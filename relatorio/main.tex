%%%%%%%%%%%%%%%%%%%%%%%%%%%%%%%%%%%%%%%%%%%%%%%%%%%%%%%%%%%%%%%%%%%%%%
% How to use writeLaTeX: 
%
% You edit the source code here on the left, and the preview on the
% right shows you the result within a few seconds.
%
% Bookmark this page and share the URL with your co-authors. They can
% edit at the same time!
%
% You can upload figures, bibliographies, custom classes and
% styles using the files menu.
%
%%%%%%%%%%%%%%%%%%%%%%%%%%%%%%%%%%%%%%%%%%%%%%%%%%%%%%%%%%%%%%%%%%%%%%

\documentclass[12pt]{article}

\usepackage{sbc-template}

\usepackage{graphicx,url}

\usepackage[brazil]{babel}   
\usepackage[utf8]{inputenc}  

\usepackage{fancyhdr}
\pagestyle{fancy}

\fancyhead[L]{ }
\fancyhead[R]{ }

%\chead{\begin{picture}(3,3) \put(-230,-15) {\includegraphics%[width=16cm, height=2.8cm, keepaspectratio=false]{logoSIRC.png}} \end{picture}}
\renewcommand{\headrulewidth}{0pt}
\sloppy
\begin{document} 


\title{Relatório Técnico Sobre o Avanço Pandemia Causada pelo Vírus SARS-CoV-2 na Cidade de Dourados, MS\\ Análise de Dados, Comparações de Cenários e Modelo Preditivo}

\author{Fernando Ferraz Ribeiro\inst{1}, Rafael H. Bordini\inst{2}, Flávio Rech
  Wagner\inst{1}, Jomi F. Hübner\inst{3} }
  


\address{Universidade Federal da Bahia -- Faculdade de Arquitetura -- LCAD
  (UFBA)\\
  Salvador, BA -- Brasil
\nextinstitute
  Department of Computer Science -- University of Durham\\
  Durham, U.K.
\nextinstitute
  Departamento de Sistemas e Computação\\
  Universidade Regional de Blumenal (FURB) -- Blumenau, SC -- Brazil
  \email{fernando.ribeiro@ufba.br, R.Bordini@durham.ac.uk,
  jomi@inf.furb.br}
}


\maketitle

\begin{abstract}
  This report presents a study on the progress of the pandemic caused by the Sars-Cov2 virus in the municipality of Dourados, MS. For this purpose, the database provided by the Ministry of Health was used. The report presents a descriptive analysis of the data available in that database. A comparison with the cases diagnosed in other municipalities in the same state and with data from the city of Manaus, AM, used as a reference for an emblematic case of the advance of the pandemic in Brazil. In the end, a predictive model adapted from DELPHI, developed by a research team linked to MIT, used as an instrument to visualize the future scenario where current trends of progress are maintained and confirmed.
\end{abstract}
     
\begin{resumo} 
  Este relatório apresenta um estudo sobre o avanço da pandemia causada pelo vírus Sars-Cov2 no município de Dourados, MS. Para tanto foi utilizada a base de dados fornecida pelo Ministério da Saúde. O relatório apresenta uma análise descritiva dos dados disponíveis na referida base. Uma comparação com os casos diagnosticados em outros municípios do mesmo estado e com os dados da cidade de Manaus, AM, utilizado como referência de um caso emblemático do avanço da pandemia no Brasil. Ao fim, um modelo preditivo adaptado do DELPHI, desenvolvido pro uma equipe de pesquisa vinculada ao MIT, utilizado como instrumento de visualização do cenário futuro onde as atuais tendências de avanço sejam mantidas e confirmadas.
\end{resumo}


\section{Introdução}

O avanço da pandemia da covid-19, causada pelo vírus SARS-CoV-2, tem se apresentado como o mais importante desafio do tempo presente. Autoridades políticas, cientistas e a sociedade tem buscado se organizar com o intuito de minimizar os grandes malefícios, direta ou indiretamente ligados à propagação desta enfermidade. De acordo com a \textit{Johns Hopkins University}, uma das principais fontes de dados mundiais sobre o tema, o número de casos confirmados no mundo já ultrapassa os 9 milhões (em 23/06/2020).

A adequada coleta e análise dos dados relativos à pandemia tem sido uma importante ferramenta no enfrentamento desta crise, sendo usada para direcionar ações, recursos e informações em todas as esferas da sociedade, na procura de um caminho menos calamitoso na lida com este nebuloso cenário.

No presente trabalho, uma análise dos dados da cidade de Dourados, MS e apresentada. A situação da cidade é comparada com os demais municípios do estado e a curva de crescimento registrada na referida cidade é comparada com a curva registrada na cidade de Manaus, AM. Em seguida os dados são utilizados para alimentar um modelo preditivo baseado no DELPHI, desenvolvido por uma equipe de pesquisadores ligada ao MIT. Na conclusão, os resultados obtidos são analisados e possíveis desdobramentos de pesquisa sugeridos. 

\section{Metodologia}\label{sec:met}

Mesmo sob a alardeada interiorização da pandemia no Brasil, o caso do município de Dourados, MS chama a atenção pelo fato de, mesmo com uma população equivalente à \(\frac{1}{4}\) da de Campo Grande, apresenta um número total de casos diagnosticados aproximadamente 60\% maior do que os registrados na capital.

A fonte de dados utilizada na elaboração deste relatório foram obtidas no portal \textbf{CORONAVÍRUS BRASIL}\footnote{https://covid.saude.gov.br/}, onde as informações oficiais do Ministério da Saúde sobre a pandemia são disponibilizados. As informações constantes como número de casos (acumulados e novos casos) e sobre a população de cada município foram extraídas desta base.

\begin{figure}[!htb]
  \centering
  \includegraphics[width=1\textwidth]{figs/casos_por_municipio.png}
  \caption{Número de casos diagnosticados por município (MS)}
  \label{fig:casosMuni}
  \end{figure}

Com base neste conjunto de dados, foram comparados o numero de casos e a população dos diferentes municípios do estado. Representadas em forma de diagramas e mapas (dados geoespaciais) procurando entender como está a distribuição de casos nas diversas regiões do estado. A curva de crescimento dos casos do município foram comparadas com a curva do município de Manaus, AM, procurando entender o quanto esta evolução se aproxima de um caso emblemático da evolução do contágio em território nacional.

\begin{figure}[!htb]
  \centering
  \includegraphics[width=1\textwidth]{figs/pop_por_municipio.png}
  \caption{População por município (MS)}
  \label{fig:popuMuniMS}
  \end{figure}


Após essas análises, os dados foram tratados e submetidos ao modelo preditivo escolhido. Ao fim, os resultados são discutidos.

\begin{figure}[!htb]
  \centering
  \includegraphics[width=1\textwidth]{figs/casos_100_mil_por_municipio.png}
  \caption{Número de casos por 100 mil habitantes em cada município (MS)}
  \label{fig:casosMuni100k}
  \end{figure}

\section{Análise de Dados}\label{sec:dados}

---

\subsection{Dados Geoespaciais}\label{ssec:geo}

\begin{figure}[!htb]
  \centering
  \includegraphics[width=1\textwidth]{figs/mapa_casos_registrados.png}
  \caption{Mapa de casos diagnosticados por município (MS)}
  \label{fig:mapaCasos}
  \end{figure}


  \begin{figure}[!htb]
    \centering
    \includegraphics[width=1\textwidth]{figs/mapa_casos_100_mil.png}
    \caption{Número de casos diagnosticados por 100 mil habitantes em cada município (MS)}
    \label{fig:mapa100K}
    \end{figure}
  

\subsection{Curva de Casos e Comparação com o cenário de Manaus}\label{ssec:curvaMAU}

\begin{figure}[!htb]
  \centering
  \includegraphics[width=.6\textwidth]{figs/Dourados_Manaus_casos.png}
  \caption{Curva do número de casos diagnosticados por 100 mil habitantes. Comparativo entre Dourados(MS) e Manaus(AM)}
  \label{fig:curva100K}
  \end{figure}
---

\begin{figure}[!htb]
  \centering
  \includegraphics[width=.6\textwidth]{figs/Dourados_Manaus_casos_log.png}
  \caption{Curva do número de casos diagnosticados por 100 mil habitantes. Comparativo entre Dourados(MS) e Manaus(AM) em escala logarítmica (eixo y)}
  \label{fig:curva100KLog}
  \end{figure}

---


\section{Modelo Preditivo}

----

\section{Conclusão}\label{conc}

---

\bibliographystyle{sbc}
\bibliography{sbc-template}

\end{document}
